\documentclass{article}
\usepackage{packages}
\begin{document}

\import{./}{title}

\frontmatter
\tableofcontents

\mainmatter
\linenumbers

{\huge{\multiTitle}}

\section*{ABSTRACT}
\emph{To be filled in later}.

\begin{refsection}

\section{Introduction}

Imagine an economy that is part of the world economy. In that economy, production and consumption activities take place, and at the intersection of that economy and the system it is embedded in

Start from different thread and show how they converge: LP theory, environmental concerns, scenario modelling, IO developments (on the international stage), LP applications for environmental analysis


This paper serves as a background paper for my SU-paper. It shall describe the RCOT, WTM, and WTM/RCOT models (including their extensions), point out commonalities, differences, characteristics, mechanisms etc.

Mention somewhere in intro that the popularity of both models (in terms of applications) warrants for an overview and investigation of certain characteristics

Perhaps include Samuelson's paper on Ricardian land \url{https://academic.oup.com/qje/article/73/1/1/1938462}

Refer to Kantorovich's original paper on LP (Kantorovich shared Leontief's view that mathematicians and production workers should collaborate, see conlcusion b), where he even argues for the minimisation of scrap, which could provide a nice link to Duchin's work \url{https://www.jstor.org/stable/2627082}. See also Koopmann's comment \url{https://www.jstor.org/stable/2627081}

Give also an overview of the earlier literature on LP-based environmental analysis, including in the LCA space, e.g. the work by Azapagic and Clift and especially by Freire \url{https://link.springer.com/article/10.1007/PL00013340} (which contains many nice references) and WIO-LP. Reference to early activity analysis (Koopmanns, Baumol etc) should be given so as to show that tradition. Regarding environmental analysis, it might be an idea to point to the Ayres-Kneese paper where they present essentially an EE-IO system but cannot solve it (the Ayres-Kneese orientation may fit well with the production function idea).

Quite a few papers also start from Kätelhöhn's model but do not give reference to D\&L (e.g. \url{https://pubs.acs.org/doi/10.1021/acssuschemeng.7b00631} or Steubing \url{https://link.springer.com/article/10.1007/s11367-015-1015-3} or Bachmann on plastic \url{https://www.nature.com/articles/s41893-022-01054-9}). Other extensions are far away from original RCOT but relevant to mention, e.g. dynamic model by Hung \url{https://onlinelibrary.wiley.com/doi/10.1111/jiec.13331#jiec13331-bib-0061}. This review of CLCA methods (incl. TCM) may also be relevant for IO applications \url{https://onlinelibrary.wiley.com/doi/10.1111/jiec.12983}, even if just to mention that other approaches exist.

Maybe have a split lit overview: some very old stuff (as a precursor to both WTM and IO-RCOT), and then the work that built on IO-RCOT and WTM respectively. Have a further distinction for LCA and IO applications (only for RCOT relevant?)

Have a semi-systematic review with more detail in the SI. Have the review grouped by applications (according to individual planetary boundaries) as well as methodological extensions.

Also, don't forget to refer to Heijungs-Suh who described pretty much SU-RCOT in words and the difficulties of rectangular outlays.

RCOT is more like an analysis technique than a causal model to explain substitution. That is, the cause for technical change is not explained, only the result of it.
WTM is a model that shows which trade relationships would be present in an idealised world (according to comparative advantage).

One topic that garnered great interest in economics during the 1950s-60s but faded away afterwards was the problem of choosing among multiple technology alternatives. This topic was dealt with theoretically under the umbrella term of \textit{choice of technique}, which brought forth various non-substitution theorems. On the empirical side, \textcite{carter1970} examined structural changes in the US economy over 30 years or so and looked specifically at technology substitutions over time. \textcite{duchin_2011} generalised Carter's model into the \textit{rectangular choice of technology} model, or RCOT.

RCOT is a linear program that allows to choose among multiple technology alternatives with homogeneous outputs. It has been theoretically explored in multiple contributions \parencite{duchin_2017,steenge_2019,duchin_2012} and applied in various studies (refs). RCOT minimises factor use while considering factor constraints and satisfying final demand.

Driven by rising environmental literacy, another topic sparked great interest at the same time. Researchers from various disciplines began to engage in efforts that came to be known as \textit{world modelling} \parencite{fontela_2004}. \textit{The Limits to Growth} by \textcite{meadows_1972} enjoyed possibly both most popularity and criticism. In the input-output community, it was Leontief's world model that became well-known \parencite{leontief_1970,leontief_1977}. 

The structure of the paper is as follows: after providing an overview of the RCOT model as formulated by \textcite{duchin_2011} in section x, we present a supply-driven counterpart and variations of it in terms of how factor exchanges are represented (section y). We then illustrate these models with a numerical example and point out key differences between the different variations (section z), after which we conclude (section zz).

Relevant for WTM (and specifically factor mobility)? \url{https://www.jstor.org/stable/1907438} Does WTM actually reach back to Arrow and his reference to an article by Chenery?
Relevant for substitution of Leontief system generally \url{https://www.sciencedirect.com/science/article/pii/0024379568900025}
Export and import prices become important in WTM (in addition to sector-specific prices and input vs output prices)
Does cross-hauling make a difference (or can even be accounted for) in WTM?

Production function as from here: \url{https://www.jstor.org/stable/23859612?seq=11} and from van den Bergh \url{https://www.jstor.org/stable/3147065?}

definitely include WIO-LP when talking about RCOT

When it comes to an overview of/ reference for WIO, use this review paper by Towa \url{https://www.sciencedirect.com/science/article/pii/S0959652619342295?via%3Dihub} and this paper to refer to the most recent dynamic WIO model \url{https://pubs.acs.org/doi/full/10.1021/acs.est.2c09676}. Mention at a later stage that the limitations of IO-RCOT also affect every WIO model and that perhaps a SU-based model might be better (because fewer limitations). this is especially interesting concerning critical materials that are typically only represented as by-products of other metal production processes.

This overview of comparative advantage literature by Belloc suggests that more needs to be included in respective models (and thus also in RCOT/WTM); especially the role of institutions is missing - only factor endowments and technological differences do not explain actual trade patterns \url{https://onlinelibrary.wiley.com/doi/abs/10.1111/j.0950-0804.2006.00274.x}. This can easily be pointed to in the qualitative production functions.
I suppose that these institutional setting don't only have an effect on trade between countries but also between industries of the same country (cartels within a country could be one special form? next to obvious legislative differences - what about standards and laws for different industries?)
Could the institutional aspect be connected to the needs-perspective. I.e. different institutional settings may affect final demand groups differently?

I might want to use the production functions to point out opportunities for further research, e.g. to include institutions, waste etc
But I should also highlight that both models are only applicable under certain assumptions (global quasi-free trade, linearity etc.) and that it might be too limited for more sophisticated analyses (e.g. concerning dynamic de-growth analyses where disequlibria and feedback effects may occur)
Is the assumption and modelling of comparative advantage a normative dilemma? I.e. if world organisation were different would comparative advantage be proven wrong?

It would be great if I could suggest an abstract model extension that allows to include considerations of e.g. insitutional setting, circularity, temporality, spatial granularity, qualitative demand differences (needs!) etc

Having capital as endowments is a problem. Switching to a Sraffian perspective with capital as produced goods may be more applicable.
Also, cross-hauling is not accounted for, is it?

Somewhere I should talk about (Keynesian) effective demand and what that means in terms of justification for both models, and then formulate a model where demand and supply are mixed (some fd and some x on the RHS).
As for (exogenous) supply (constraints): these can either be defined as a constraint, or supply and fd may be swapped for some goods, or the supply constraint can be introduced when turning the LCP into an MCP (As per here \url{https://juliabook.chkwon.net/book/complementarity})

In the neoclassical understanding, an economy is generally represented as the set $E_{1} = (\phi, V, U, e)$ (cf. Shestalova's book chapter and cited references therein). $E_{1}$ is further characterised by all prices, i.e. both commodity and all sorts of factor prices being endogenous. In contrast, in Duchin's original, sustainable development-oriented models, an economy is instead represented as $E_{2} = (\pi, \phi, I, A, e)$, where the input and output matrices are based on pseudo-single production and where an exogenous factor mark-up reflects the precautionary principle. Kätelhöhn et al. and Koslowski et al. have expanded Duchin's rectangular choice model to cover joint production instead, resulting in an economy described as $E_{3} = (\pi, \phi, V, U, e)$; we see that also here a hybrid factor price is employed, with $\pi$ being the exogenous portion.

Classify the model according to Rose's scheme

When reviewing the TCM-based papers, I should add that the WTM formalism can also be applied to TCM-based approaches as origin-destination pairs also play a role in LCA.

For desirable factor exchanges (e.g. employment), the factor constraint must/should be an equality. Does it thus make sense to distinguish between desirable and undesirable factors exchanges? Equalities for desirable ones, inequalities for undesirable ones?
Maybe propose to have actually three factor equations/inequalities (<=>); not only to distinguish between desirable and undesirable factor exchanges, but also between in and outputs of factors

Not sure if this PIK paper on a new CBA-CEA/ Welfare economics is relevant as background \url{https://www.degruyter.com/document/doi/10.1515/jbnst-2022-0022/html#j_jbnst-2022-0022_ref_025}

Maybe include earlier LCA-related work on LP models, such as Freire \url{https://link.springer.com/article/10.1007/PL00013340} and references therein (especially Azapagic!), or Dyckhoff \url{https://doi.org/10.1016/S0377-2217(00)00154-5} and references therein as well as \url{https://link.springer.com/article/10.1007/s11573-017-0885-1}; not sure what to make of this one \url{https://link.springer.com/article/10.1007/s11573-020-01004-x}

Have a bit more on the WTMBT and the possibility to derive scenario-based MRIOTs from it.

\section{Historical background}

Describe the origin (both historically and conceptually) of RCOT+WTM -> world modelling + comparative advantage (Ricardo)

Both models are based on the same underlying mechanism, comparative advantage, yet are used for very different purposes. Let us now examine how we got to this point.

\section{Model overview}

Notation. Plain introduction to the primal and dual models of both WTM and RCOT.

As we will describe later, the WTM and RCOT depend on the same underlying mechanism which is based on the principle of comparative advantage.
The most parsimonious formulation is the WTM/RCOT model presented in \textcite{duchin_2012}, where the primal reads as follows:

\begin{mini}
    {\bm{x}^{*}}{\sum_{r} \bm{\pi}^{\prime}_{r} \bm{R}^{*}_{r} \bm{x}^{*}_{r}}{\label{eq:IOT_RCOT_quant}}{Z_{IO}=}
    \addConstraint{\sum_{r} (\bm{I}^{*}_{r} - \bm{A}^{*}_{r}) \bm{x}^{*}_{r}}{= \sum_{r} \bm{y}_{r}}{}
    \addConstraint{\bm{R}^{*}_{r} \bm{x}^{*}_{r}}{\le \bm{\varphi}_{r}}{\quad \forall r}
    \addConstraint{\bm{x}^{*}_{r}}{\ge \bm{0}}{\quad \forall r,}
\end{mini}

and the dual as:

\begin{maxi}
    {\bm{p}, \bm{r}_{r}}{\sum_{r} \bm{y}^{\prime}_{r} \bm{p} - \bm{\varphi}^{\prime} \bm{r}}{\label{eq:IOT_RCOT_price}}{W_{IO}=}
    \addConstraint{(\bm{I}^{*}_{r} - \bm{A}^{*}_{r})^{\prime} \bm{p} - {{\bm{R}}^{*}_{r}}^{\prime} \bm{r}_{r}}{\le {{\bm{R}}^{*}_{r}}^{\prime} \bm{\pi}_{r}}{\quad \forall r}
    \addConstraint{\bm{p}, \bm{r}_{r}}{\ge \bm{0}}{\quad \forall r.}
\end{maxi}



While RCOT is a more parsimonious and general version of Carter's square COT model, WTM is more or less directly based on square COT in the case of a free trade world.

Describe the topics of feasibility and shared factors etc.

\begin{itemize}
    \item characteristics
    \item examination of feasibility criteria
    \item Lagrange multipliers
    \item justification of ixi/cxc tables
    \item differentiation to sector disaggregation
    \item altered calculation of measures such as total factor productivity (and possibly also of circularity indicators \url{https://onlinelibrary.wiley.com/doi/10.1111/jiec.13446}. I suppose that all of that would simply be scenario-based and, if not directly, could be derived at least when doing the WTMBT (not sure, though))
\end{itemize}

\section{Review}

Review past applications and how they might have extended the model framework.

Describe how both WTM and RCOT are generally to be applied (scenario-modelling etc) and then actual published applications.

\section{Interpretations \& observations}

In this section, we want to further characterise the models and provide additional interpretations. We hope that other researchers may find this helpful in the case of them considering using any of the models or in terms of potential avenues for further research.

Interpret both models

0. Both RCOT and WTM as a game between man and nature (Morton \url{https://www.jstor.org/stable/2549610?origin=crossref&seq=13}); relationship of these games (minimax) to precationary principle (maximin) as per \url{https://onlinelibrary.wiley.com/doi/epdf/10.1111/j.1467-9760.2006.00237.x}; be careful as to the interpretation of the principle (recall Edgar's comment on (un)employment in such models); refer to \url{https://benjaminbrooks.net/downloads/brooksreny_equivalence.pdf} and \url{https://link.springer.com/article/10.1007/s00182-012-0328-8} and the printed paper (was it this one? \url{https://onlinelibrary.wiley.com/doi/abs/10.1111/j.1477-9552.1969.tb01357.x}); plus link potentially to tragedy of commons (perfectly mobile factors as "common pool resources")? \\
0x. Distinction between tables in monetary, physical, or hybrid units. Which constructs can be applied? Interpretation of price model as prices vs price indexes.\\
1. Role of comparative advantage - both models are almost identical in the sense that in both of them an alternative way of producing something is available; the interpretations of this mechanism are however very different \\
2. A different interpretation of comparative advantage: Compare Duchin's model with ten Raa's/Shestalova's. Formulate Duchin's model as an MCP to outline the differences (assumed baseline factor prices vs final demand volume variable (ten Raa postulates that knowledge about consumer preferences is necessary, e.g. here \url{https://journals.openedition.org/oeconomia/1833}) + ten Raa's sector-specific factor prices). All boiling down to a different (future sustainability) world view where all factors cost something (this could be the general theme of the paper); ten Raa maximises profit in Leontief's trade model while Duchin maximises profit less scarcity payments \url{https://pure.uvt.nl/ws/portalfiles/portal/969574/neoclass.pdf}\\
3. Comparative advantage and NSTs: the role of number of factors, fully-employed factors, and shared vs tech-specific factors \\
4. Comparative advantage vs hypothetical extraction: an (additive) inverse problem? / two sides of the same coin / a primal and dual perspective; comp-adv models assume an additional technology to be available whereas HEM assumes that a chosen technology is not available (with also here very different purposes: disaster impact and material pressures vs trade and tech choice) \\
5. Comparative advantage models and ecological economics: describe the role of substitution, with an emphasis on factor substitution and weak sustainability; comparative advantage models are not proper ecolecon models but fall into the pragmatic environmental econ corner \url{https://www.clivespash.org/wp-content/uploads/2015/04/2013_Spash_EE_Shallow_or_Deep.pdf}; is the comp-adv relationship a point to strengthen a classical interpretation of IO or at least these models here? (ten Raa's model of comp-adv. is even neoclassical...); refer also to Baumgärtner's joint production paper

Perhaps also a note on the similarity of RCOT and WIO-LP?
Link to Ayres\&Kneese; is material balance principle adhered to (in any unit-formulation?)
Does it make sense to connect to Daly's qualitative pollution input-output model?
Could Dyckhoff's paper be interesting for discussion on bads? \url{https://publications.rwth-aachen.de/record/949107/files/949107.pdf}

Highlight that all of above interpretations refer to RCOT / WTM as originally presented by Duchin; joint production concerns can then of course also be interpreted via SU-RCOT

It might be useful to explore the role of commodity prices more e.g. with reference to Guillaume's paper \url{https://www.sciencedirect.com/science/article/pii/S0921800916302269#s0005}

I think I should highlight the role of price homogeneity even more, e.g. with this paper by Algarin \url{https://onlinelibrary.wiley.com/doi/10.1111/jiec.12502#jiec12502-bib-0015}. Although not mentioned there, be reminded of Dietzenbacher's comment \url{https://sci-hub.se/https://www.sciencedirect.com/science/article/abs/pii/S0921800905002119?via%3Dihub#bib13} on sub-sector flows and according prices, thus pointing to the aggregation problem outlined by Guillaume.

Pack all of the material on factor mobility here.

The same for shared responsibility and Coase and Pigou.

Point out that RCOT-WTM fits generally very well to model not only planetary boundaries but doughnut economy (put into section on matrix games?). Just a difficulty to find social categories that are dependent of activity levels. Anyway, just highlight and give also example of needs-based perspective (at least by disaggregating final demand by need-categories)

Note that factors don't have to have an ex-ante price. In that case, it simply means that this factor is of no importance to society (no precautionary action!). That factor might however have to be priced if in scarce supply - then, the price would be determined purely by the scarcity rent. That hybrid nature of the price allows to corroborate the theory of comparative advantage, where also originally unpriced factors may change international trade - simply by transgressing biophysical boundaries.

On the topic of factor mobility: It seems that the exclusion of non-tradeables (matrix J) in ten Raa is dealt with very similiarly \url{https://www.tandfonline.com/doi/epdf/10.1080/09535319100000012?needAccess=true}. Include also the very recent paper by Dilekli \url{https://www.tandfonline.com/doi/full/10.1080/09535314.2023.2272213}

HO model cannot be used for deriving commodity prices as it assumes given commodityprices. This model differs therefore starkly from the others. Unless one starts with a monetary SUT/IOT, it is (I think) virtually impossible to determine commodity prices. The focus of the model is thus clearly on the actual output under imposed/given prices; isn't that problematic even in trade considerations? It certainly is when it comes to national modelling, because... how are the prices determined/made up?

ten Raa writes that exceptional economies are not interesting because they can "either produce nothing or anything". In RCOT, I think such anything-economies are not possible - at one point an infeasibility always kicked in. But NEO is pretty close to that - although he uses the NEO problem to show non-substitution; don't quite understand... , \url{https://www.sciencedirect.com/science/article/abs/pii/S0022053185710629?via%3Dihub}

Johansen writes that no additional assumptions are imposed on a standard linear IO model. But what really is a standard IO model? Where is it described that factor prices have to be endogenous?

Need to have a separate section on economic rent and what is meant by it, i.e. which production factors can I actually include at all where the concept of rent becomes intuitive?

This LCA review on technical substitutability could be interesting to add. They look at how the substitution of products (e.g. primary vs secondary metals) is modelled in LCA. Can product substitution be modelled with RCOT? \url{https://www.sciencedirect.com/science/article/pii/S0956053X23005482?dgcid=author}

Does RCOT follow the "biophysical theory of  value", as discussed by Patterson? Or how does it relate to it? That is, can I classify the RCOT dual in some way? It seems that it is pretty close to Costanza and Hannon's approach... \url{https://doi.org/10.1016/S0921-8009(97)00166-3} + when combined with waste, is it similar to my suggested model? Also, maybe RCOT requires the explicit definition of a numeraire?

Maybe it's easier to classify the role of value/price in RCOT using figure 2 and table 1 from here \url{https://www.sciencedirect.com/science/article/pii/S0921800922000441} Also this article might be helpful: \url{https://www.sciencedirect.com/science/article/pii/S0921800919308651}

Have a good deal about the early work of LCA-LP and how RCOT/WTM relates to it

If subsidies and taxes are included as factors: Would subsidies (or negative TLS) result in a joint-produciton setting? That is, in IO-RCOT, only one element per column is positive (in the constraint matrix) whereas in SU-RCOT, several can be. Factors are all negative (because of inequality sign), but subsidies would be positive. Or can that be alleviated by only saying that subsidies have a negative sign but that the corresponding factor price is +/-1?

Perhaps a useful intro to factor mobility with factor examples: \url{https://www.sciencedirect.com/science/article/pii/S0166046205000074?casa_token=wTqO3r8ABRQAAAAA:BwytLvQMs_zPrgQbELWkEUPYzZ0TcBGdPUm3nuxLlrrcPBdkdibJH3YKNS3h1E9RZWdVBJe6bQ}

\subsection{Shifting production factors}

When factors are shared by multiple technologies, both existing and/or alternative ones, it is worth considering the aspect of factor mobility. Although the term is most commonly applied in the context of trade economics, we use it here to refer to the (im)mobility of production factors between industries. Why might this be relevant in the RCOT context? 

Let us first consider the empirical base setting without alternative technologies ($t = k$) and where the employed factors are assumed to be the actual factor limits $\bm{f}^{\circ} \le \bm{f}$; i.e. denote for notational simplicity $\bm{\Phi} = (\bm{F} \lor \bm{R} \widehat{\bm{x}^{\circ}})$ (depending on if $i \times i$ or $c \times c$ setting) such that $\bm{\Phi i} = \bm{\phi} = \bm{f}^{\circ}$ and where $\bm{x}^{\circ}$ is the empirically observed total output while $\bm{x}$ is the total output resulting from the LP. Further, let us introduce the $p \times k$ factor mobility matrix $\bm{P}$ which shall define which factors can move between which sectors. Let $\bm{R} \ast \bm{P}$ be the Khatri-Rao product, or block-wise Kronecker product of matrices $\bm{R}$ and $\bm{P}$. Then, in the case of perfect factor mobility, one may represent the system's factor constraints as $(\bm{R} \ast \bm{P}) \bm{x} = \bm{R} \bm{x} \le \bm{f}^{\circ} = (\bm{R} \ast \bm{P}) \bm{x}^{\circ} = \bm{R} \bm{x}^{\circ}$, where $p = v + r$ and where the rows of $\bm{P}$ consist of row unit vectors $\bm{i}'$, one for each factor. In contrast to that, perfect immobility of all factors may be represented as $(\bm{R} \ast \bm{P}) \bm{x} \le (\bm{R} \ast \bm{P}) \bm{x}^{\circ}$, where $p = k \times (v + r)$ and where each $k \times k$ row block contains an identity matrix $\bm{I}$. Alternatively, one may express the latter case as $\bm{R} \widehat{\bm{x}} \le \bm{\Phi} = \bm{R} \widehat{\bm{x}^{\circ}}$ where the inequality applies now element-wise. Perfect immobility of all factors is thus the situation when all considered factors are technology-specific.

The original formulation by \textcite{duchin_2011} models primarily perfectly mobile factors, i.e. where all factors can move freely between industries as per the factor constraint $\bm{\phi}^{*} = \bm{R}^{*} \bm{x}^{*} \le \bm{f}$ shown in model \ref{eq:IOT_RCOT_quant}\footnotemark{}. One may easily represent this constraint with the same formalism as just seen, now only extended for the respective technology alternatives: $(\bm{R}^{*} \ast \bm{P}^{*}) \bm{x}^{*} = \bm{R}^{*} \bm{x}^{*} \le \bm{f}$ where $\bm{P}$ is of shape $(v + r) \times t$.

In another context and for another model, the case of perfect factor immobility was highlighted by, among others, \textcite{tenRaa_1994,shestalova_2019}; it describes a situation when all factors cannot or shall not move between industries. As seen above, this is straightforward to model when the available factors are fully employed by the given technologies where no additional technology alternatives are present. However, once our original system turns into the rectangular one $(\bm{A}^{*}, \bm{I}^{*}, \bm{R}^{*})$, we struggle to define a factor constraint equivalent to the one above because of the dimension mismatch between left-hand and right-hand sides (LHS and RHS) of the inequality. We suggest two ways to deal with this issue and explain them, for the sake of generality and the present context of technology choice, for the case when we set the allowed factor use to the actually available factor endowments $\bm{f}$.

\footnotetext{One may speak in this case of economy-wide factor pools.}

Targeting the RHS of the factor constraint, the first approach is to allocate each available factor according to some rule to each technology, both old and new, thus creating technology-specific factor budgets. Let us introduce the $q \times 1$ block-wise factor allocation vector $\bm{\eta}^{*}$ for that purpose where each block sums to 1 and be $\bm{\eta}^{*} \ast \bm{f}$ the Khatri-Rao product of $\bm{\eta}^{*}$ and $\bm{f}$. If we allocate each factor to each technology, $\bm{\eta}^{*}$ be of row-dimension $q = (v + r) \times t$ and where each element of each $t \times 1$ row block allocates a specified share of the total factor budget to each technology. The resulting factor constraint for perfect factor immobility is $\bm{R}^{*} \bm{x}^{*} \le \bm{\eta}^{*} \ast \bm{f}$. If one were to apportion factors to technology groups instead of only individual technologies, the row dimension of $\bm{\eta}^{*}$ would change accordingly to $q = (v + r) \times \tau$ where $k \le \tau \le t$; one will have to adjust the LHS appropriately.

Alternatively, manipulating the LHS of the factor constraint, i.e. $\bm{\phi}^{*}$, one may choose to aggregate the to-be-modelled factor use of specified technologies such that factors are mobile between $\rho$ chosen sets of technologies, here between a technology and its alternative, but immobile between the rest. In this case, one may formulate the factor constraint similar to earlier as $(\bm{R}^{*} \ast \bm{P}^{*}) \bm{x}^{*} \le \bm{f}$ but where $\bm{P}^{*}$ is now of shape $((v + r) \times (\rho + 1)) \times t$ and where each $(\rho + 1) \times t$ row block sums along its rows to unity $\bm{i}'$. If one aggregates the factor use of more than the originally dominant technologies and their respective alternatives such that the row dimension of $\bm{\phi}^{*}$ becomes less than $(v + r) \times k$, one has to adjust the RHS of the factor constraint accordingly. Clearly, the reduction of factor mobility to selected technology sets does not impose perfect factor immobility as strictly on the system as the previous approach, and reallocation of production enabled by shared factors between the chosen technologies becomes a necessary possibility.

These two approaches allow to impose (almost) perfect factor immobility between all technologies available in the extended system. This stands in contrast to the perfect mobility of all factors seen earlier and which increases allocative efficiency \parencite{amores_2014}. A middle-ground between both extremes can be found when one chooses to impose selective, partial mobility of factors; selective in the sense that multiple technologies that are not necessarily only alternatives to each other share factor pools; and partial in the sense that multiple technology groups may share common factor pools to differing degrees. Both can be easily accommodated for with the allocation and mobility matrices introduced above. For the former, by allowing the technology sets $\rho$ to be filled with any technology combination (not only the dominant technology and its alternatives being grouped together), and for the latter by allowing elements of the $\bm{P}^{*}$-matrix to take on values other than zero or one as long as the column sums of each row block equal unity, while at the same time ensuring that the available factors are apportioned accordingly through a sensible use of $\bm{\eta}^{*}$. Additionally, one may allow for different allocation patterns for different factors, instead of the homogeneous way presented here. Importantly, however, the row dimensions of LHS and RHS of the factor constraint must match. It is worthwhile mentioning at this point that any deviation away from assuming perfect factor mobility as expressed in the IO-RCOT model may require the adjustment of not only the factor constraint but also other parts of the primal and dual; for example, one may choose to formulate the primal's objective function as \{$\underset{x^{*}}{\rm minimize} \quad \bm{i}' (\Bar{\bm{\pi}}' \ast {\bm{P}^{*}}^{\prime}) (\bm{R}^{*} \ast \bm{P}^{*}) \bm{x}^{*} $\} in the case of perfect factor immobility. Moreover, while any of these approaches may help to model between which sectors the factors can move and to which degree, we have not set any measure of their stickiness, i.e. how easily factors move from one sector to another.\footnotemark{}

\footnotetext{Somewhat in relation to sector-specific factor constraints, one may also choose to consider sector-specific factor prices $\Bar{\bm{\Pi}}$ (or $\Bar{\bm{\Pi}}^{*}$ if technology alternatives are added). In that case, both primal and dual would have to be adjusted accordingly, e.g. the primal's objective function would turn into $\bm{i}' \Bar{\bm{\Pi}}^{* \prime} \odot \bm{R}^{*} \bm{x}^{*}$, where $\odot$ represents the element-wise Hadamard product.}

To counter this problem, and thus widen the problem space, one could try to implement some measure of inertia. This could be done by including an additional objective function, thus resorting to multi-objective optimisation \parencite[for a review of multi-objective LP in IO, see ][]{oliveira_2014}. This additional objective function could, similar to the well-known RAS-procedure, minimise the system difference, i.e. the difference between the empirical base state (with or without active factor constraints and without alternative technologies) and the new state where either new technology alternatives were introduced, factor constraints became active, or both. The new objective function would hence be \{$ {\rm minimize} \quad (h_{1}(x^{*}), h_{2}(x^{*}))$\}, where $h_{1}(x^{*})$ is the original objective function and where $h_{2}(x^{*})$ sets the two system states $(\bm{S}^{*} \ast \bm{P}^{*}) \bm{x}^{*} \le (\bm{S} \ast \bm{P}) \bm{x}^{\circ}$ into relation, with the factor mobility matrices being of appropriate shape.

The exact shape of $h_{2}(x^{*})$ is, similar to any of the functions employed for the RAS-procedure \parencite{jackson_2004}, subject to individual considerations; then, when weighting the state difference, an implicit factor inertia is introduced (by trying to keep the differences between the systems small). As for the combination of both objective functions, generally, if an \textit{a priori} solution method is chosen, a certain preference between the two objective functions has to be defined - which serves as a proxy for determining the stickiness of factors. It should be noted that the economic interpretation of the LP is changed once the additional objective function is introduced, and that the formulation of a dual is thus also complicated. The possibility of dealing with the problem of factor mobility and stickiness is only insinuated here but deserves a more focused examination elsewhere, especially if one wants to model it explicitly.

As regards the factor mobility of capital goods, one may be moreover reminded of \citeauthor{carter1970}'s (1970, chapter 12) discussion of the embodiment hypothesis and putty-clay models which certainly applies to the more parsimonious RCOT framework as well. In fact, as IO-RCOT is formulated now, with perfectly mobile factors, it allows for both ex-ante and ex-post flexibility. Lock-in effects due to (partially) immobile factors implying ex-post rigidity are, however, an empirical reality which mandate the sensible modification of the factor constraint and an appropriate choice of time horizon depending on the purpose of the study.

As for that choice of time horizon: if the modeller's purpose is to evaluate only the \textit{hypothetical} competitiveness (or efficiency) of emerging, obsolete, or hypothetical vs dominant technologies, i.e. if technology $t_{A}$ would be more efficient than technology $t_{B}$ in terms of factor exchanges and factor cost reduction, then the limitations posed by factor immobility can be disregarded and the analysis can be undertaken for the chosen accounting period in past, present, and future.\footnotemark{} If the purpose is, however, to construct a future scenario to assess if an emerging (or entirely hypothetical) technology could \textit{actually} substitute a dominant technology, factor (im)mobility and stickiness become important: if the time horizon is too close to the present, it is questionable whether certain factors like labour force and capital goods could shift or be set up fast enough to allow substitution; if the modelled time period lies far in the future, the modelling may perhaps not be limited by the problem of factor stickiness but rather by the availability of sensible data for the technological recipes of both originally dominant and new technologies. On a side note, if the modeller's purpose is to perform either of the above for the case where the supposedly new technology is already incumbent, i.e. it is part of a dominant sector, the latter's technological recipe must be split and adjusted. Concluding, if the study's aim is not a hypothetical efficiency analysis, a more realistic, explicit modelling of factor mobility and stickiness is warranted, potentially requiring a dynamic model formulation.

\footnotetext{Provided appropriate data is available, one may choose to perform this analysis not only for the typical period of a year but any arbitrary period.}

\subsection{Shared responsibility for factor exchanges}

Next, we want to establish a connection between the treatment of factor mobility in the present framework, the concept of shared responsibility, and the Coase theorem. The literature on shared responsibility addresses the problem of how to attribute factor exchanges equitably to actors along supply chains \parencite[see for example][]{lenzen_2007}. The Coase theorem \parencite[based on][]{coase_1960} postulates that, at zero transaction costs, the allocation of factors ($\bm{R} \widehat{\bm{x}}$ in our notation) is Pareto efficient and independent of the allocation of property rights (or structure of entitled factor mobility $(\bm{R} \ast \bm{P})$, in our context). While the consequences at the individual level may sometimes be perceived as unfair, thus questioning the optimality of factor allocation and making it tempting to attribute proportionally greater responsibility to a few selected actors, this view changes at the grander scale: due to its interconnectedness, Steenge (\textbf{REF: Some old book chapter cited in Lenzen}) argues, the entire economy is effectively responsible for the factor exchanges occurring within it, such that each sector can be attributed a proportional share of the blame, provided that the economy is indecomposable. This insight is to be supplemented with the result by \textcite{amores_2014} that factor mobility increases allocative efficiency.

Taken together, this means that for two situations of differently allocated property rights, one may be more efficient than the other, even though the factor allocation in each of them is Pareto efficient. Put differently: An economy employs the most efficient combination of available technologies in any setting of constrained or unconstrained factor mobility and constrained factor availability, but the case of perfect factor mobility spares more resources while achieving a higher utility than any other setup. It becomes then interesting to consider the reasons for (partial) factor immobility. When a factor cannot move between sectors due to structural conditions, e.g. biophysical limits to land conversion, the originally chosen factor allocation remains in any case optimal. When, however, a factor cannot move because of arbitrary or otherwise easily removable limitations, e.g. sectoral emission budgets, the system's allocative efficiency is put into question: if the respective factor can be made mobile without incurring transaction costs, the factor allocation was not optimal and disproportionally greater responsibility may be attributed to individual actors; if no such adjustment is possible or only at non-zero transaction costs, the factor allocation was optimal and the responsibility may be split proportionally.

How does that look when a sector can choose between multiple technology alternatives? Due to the nature of IO-RCOT as an LP, we know that it minimises factor costs and as such aims at an optimal allocation of factor exchanges. Transaction costs are not explicitly represented. If all factors are perfectly mobile between technologies, the system's allocation of factors is optimal, irrespective of the occurrence of substitution between technology alternatives. If a factor is limited in its mobility between a given sector's technology alternatives or in its economy-wide mobility, the allocation's optimality depends yet again on the underlying reason for the respective factor's reduced mobility.

All these considerations are, of course, subject to the limitations posed by the choice of time horizon discussed earlier. Realistically, removing boundaries to a factor's mobility would incur transaction costs, thus putting the factor allocation's (and system's) optimality into question. At the same time, introducing new technology alternatives may require an arbitrary choice as to the proportion and speed of mobility of its required factors. Either way, it seems that IO-RCOT in its current form is not entirely adequate to account for losses due to transaction costs, modification of factor mobility, overcoming of factor stickiness, and responsibility attribution, all of which is necessary to evaluate if the system's technology choice is optimal over time.

\section{The road ahead}

Although the focus of Duchin's models lies on technological change, one shall not lose sight of other required changes in socio-ecological systems. Technofixes alone will hardly solve the many issues that the global socio-economic metabolism has created in recent history \url{https://link.springer.com/article/10.1007/s10098-002-0173-8}. Refer back to tragedy of the commons and similar discussions. Also refer to Duchin's other work, showing that she DID think about these issues. (Not sure how much she covered for example here \url{https://global.oup.com/academic/product/the-future-of-the-environment-9780195085747?cc=no&lang=en&}.) In that sense, further extensions of these rectangular generalisation may allow to broaden the applicability (to e.g. ...)


\newrefcontext[sorting=nyt] % sort the paper by name, year, title
\printbibliography[heading = bibintoc] % 'bibintoc' inserts our bibliography into the table of contents

\end{refsection}

%%%%%%%%%%%%%%
% Appendix
% Inserting appendix with separate settings
\newpage
\setcounter{page}{1}
\renewcommand{\thepage}{A-\arabic{page}}
\linenumbers*
\addappendix

%Reset numbering of tables and equations in appendix, starting with A.
\renewcommand{\thetable}{A.\arabic{table}}
\setcounter{table}{0}
\renewcommand{\theequation}{A.\arabic{equation}}
\setcounter{equation}{0}

\begin{refsection}
\addcontentsline{toc}{section}{Appendix A - old intro notes}
\section*{Appendix A - old intro notes}
Imagine an economy that is part of the world economy. In that economy, production and consumption activities take place, and at the intersection of that economy and the system it is embedded in

Start from different thread and show how they converge: LP theory, environmental concerns, scenario modelling, IO developments (on the international stage), LP applications for environmental analysis


This paper serves as a background paper for my SU-paper. It shall describe the RCOT, WTM, and WTM/RCOT models (including their extensions), point out commonalities, differences, characteristics, mechanisms etc.

Mention somewhere in intro that the popularity of both models (in terms of applications) warrants for an overview and investigation of certain characteristics

Perhaps include Samuelson's paper on Ricardian land \url{https://academic.oup.com/qje/article/73/1/1/1938462}

Refer to Kantorovich's original paper on LP (Kantorovich shared Leontief's view that mathematicians and production workers should collaborate, see conlcusion b), where he even argues for the minimisation of scrap, which could provide a nice link to Duchin's work \url{https://www.jstor.org/stable/2627082}. See also Koopmann's comment \url{https://www.jstor.org/stable/2627081}

Give also an overview of the earlier literature on LP-based environmental analysis, including in the LCA space, e.g. the work by Azapagic and Clift and especially by Freire \url{https://link.springer.com/article/10.1007/PL00013340} (which contains many nice references) and WIO-LP. Reference to early activity analysis (Koopmanns, Baumol etc) should be given so as to show that tradition. Regarding environmental analysis, it might be an idea to point to the Ayres-Kneese paper where they present essentially an EE-IO system but cannot solve it (the Ayres-Kneese orientation may fit well with the production function idea).

Quite a few papers also start from Kätelhöhn's model but do not give reference to D\&L (e.g. \url{https://pubs.acs.org/doi/10.1021/acssuschemeng.7b00631} or Steubing \url{https://link.springer.com/article/10.1007/s11367-015-1015-3} or Bachmann on plastic \url{https://www.nature.com/articles/s41893-022-01054-9}). Other extensions are far away from original RCOT but relevant to mention, e.g. dynamic model by Hung \url{https://onlinelibrary.wiley.com/doi/10.1111/jiec.13331#jiec13331-bib-0061}. This review of CLCA methods (incl. TCM) may also be relevant for IO applications \url{https://onlinelibrary.wiley.com/doi/10.1111/jiec.12983}, even if just to mention that other approaches exist.

Maybe have a split lit overview: some very old stuff (as a precursor to both WTM and IO-RCOT), and then the work that built on IO-RCOT and WTM respectively. Have a further distinction for LCA and IO applications (only for RCOT relevant?)

Have a semi-systematic review with more detail in the SI. Have the review grouped by applications (according to individual planetary boundaries) as well as methodological extensions.

Also, don't forget to refer to Heijungs-Suh who described pretty much SU-RCOT in words and the difficulties of rectangular outlays.

RCOT is more like an analysis technique than a causal model to explain substitution. That is, the cause for technical change is not explained, only the result of it.
WTM is a model that shows which trade relationships would be present in an idealised world (according to comparative advantage).

One topic that garnered great interest in economics during the 1950s-60s but faded away afterwards was the problem of choosing among multiple technology alternatives. This topic was dealt with theoretically under the umbrella term of \textit{choice of technique}, which brought forth various non-substitution theorems. On the empirical side, \textcite{carter1970} examined structural changes in the US economy over 30 years or so and looked specifically at technology substitutions over time. \textcite{duchin_2011} generalised Carter's model into the \textit{rectangular choice of technology} model, or RCOT.

RCOT is a linear program that allows to choose among multiple technology alternatives with homogeneous outputs. It has been theoretically explored in multiple contributions \parencite{duchin_2017,steenge_2019,duchin_2012} and applied in various studies (refs). RCOT minimises factor use while considering factor constraints and satisfying final demand.

Driven by rising environmental literacy, another topic sparked great interest at the same time. Researchers from various disciplines began to engage in efforts that came to be known as \textit{world modelling} \parencite{fontela_2004}. \textit{The Limits to Growth} by \textcite{meadows_1972} enjoyed possibly both most popularity and criticism. In the input-output community, it was Leontief's world model that became well-known \parencite{leontief_1970,leontief_1977}. 

The structure of the paper is as follows: after providing an overview of the RCOT model as formulated by \textcite{duchin_2011} in section x, we present a supply-driven counterpart and variations of it in terms of how factor exchanges are represented (section y). We then illustrate these models with a numerical example and point out key differences between the different variations (section z), after which we conclude (section zz).

Relevant for WTM (and specifically factor mobility)? \url{https://www.jstor.org/stable/1907438} Does WTM actually reach back to Arrow and his reference to an article by Chenery?
Relevant for substitution of Leontief system generally \url{https://www.sciencedirect.com/science/article/pii/0024379568900025}
Export and import prices become important in WTM (in addition to sector-specific prices and input vs output prices)
Does cross-hauling make a difference (or can even be accounted for) in WTM?

Production function as from here: \url{https://www.jstor.org/stable/23859612?seq=11} and from van den Bergh \url{https://www.jstor.org/stable/3147065?}

definitely include WIO-LP when talking about RCOT

When it comes to an overview of/ reference for WIO, use this review paper by Towa \url{https://www.sciencedirect.com/science/article/pii/S0959652619342295?via%3Dihub} and this paper to refer to the most recent dynamic WIO model \url{https://pubs.acs.org/doi/full/10.1021/acs.est.2c09676}. Mention at a later stage that the limitations of IO-RCOT also affect every WIO model and that perhaps a SU-based model might be better (because fewer limitations). this is especially interesting concerning critical materials that are typically only represented as by-products of other metal production processes.

This overview of comparative advantage literature by Belloc suggests that more needs to be included in respective models (and thus also in RCOT/WTM); especially the role of institutions is missing - only factor endowments and technological differences do not explain actual trade patterns \url{https://onlinelibrary.wiley.com/doi/abs/10.1111/j.0950-0804.2006.00274.x}. This can easily be pointed to in the qualitative production functions.
I suppose that these institutional setting don't only have an effect on trade between countries but also between industries of the same country (cartels within a country could be one special form? next to obvious legislative differences - what about standards and laws for different industries?)
Could the institutional aspect be connected to the needs-perspective. I.e. different institutional settings may affect final demand groups differently?

I might want to use the production functions to point out opportunities for further research, e.g. to include institutions, waste etc
But I should also highlight that both models are only applicable under certain assumptions (global quasi-free trade, linearity etc.) and that it might be too limited for more sophisticated analyses (e.g. concerning dynamic de-growth analyses where disequlibria and feedback effects may occur)
Is the assumption and modelling of comparative advantage a normative dilemma? I.e. if world organisation were different would comparative advantage be proven wrong?

It would be great if I could suggest an abstract model extension that allows to include considerations of e.g. insitutional setting, circularity, temporality, spatial granularity, qualitative demand differences (needs!) etc

Having capital as endowments is a problem. Switching to a Sraffian perspective with capital as produced goods may be more applicable.
Also, cross-hauling is not accounted for, is it?

Somewhere I should talk about (Keynesian) effective demand and what that means in terms of justification for both models, and then formulate a model where demand and supply are mixed (some fd and some x on the RHS).
As for (exogenous) supply (constraints): these can either be defined as a constraint, or supply and fd may be swapped for some goods, or the supply constraint can be introduced when turning the LCP into an MCP (As per here \url{https://juliabook.chkwon.net/book/complementarity})

In the neoclassical understanding, an economy is generally represented as the set $E_{1} = (\phi, V, U, e)$ (cf. Shestalova's book chapter and cited references therein). $E_{1}$ is further characterised by all prices, i.e. both commodity and all sorts of factor prices being endogenous. In contrast, in Duchin's original, sustainable development-oriented models, an economy is instead represented as $E_{2} = (\pi, \phi, I, A, e)$, where the input and output matrices are based on pseudo-single production and where an exogenous factor mark-up reflects the precautionary principle. Kätelhöhn et al. and Koslowski et al. have expanded Duchin's rectangular choice model to cover joint production instead, resulting in an economy described as $E_{3} = (\pi, \phi, V, U, e)$; we see that also here a hybrid factor price is employed, with $\pi$ being the exogenous portion.

Classify the model according to Rose's scheme

When reviewing the TCM-based papers, I should add that the WTM formalism can also be applied to TCM-based approaches as origin-destination pairs also play a role in LCA.

For desirable factor exchanges (e.g. employment), the factor constraint must/should be an equality. Does it thus make sense to distinguish between desirable and undesirable factors exchanges? Equalities for desirable ones, inequalities for undesirable ones?
Maybe propose to have actually three factor equations/inequalities (<=>); not only to distinguish between desirable and undesirable factor exchanges, but also between in and outputs of factors

Not sure if this PIK paper on a new CBA-CEA/ Welfare economics is relevant as background \url{https://www.degruyter.com/document/doi/10.1515/jbnst-2022-0022/html#j_jbnst-2022-0022_ref_025}

Maybe include earlier LCA-related work on LP models, such as Freire \url{https://link.springer.com/article/10.1007/PL00013340} and references therein (especially Azapagic!), or Dyckhoff \url{https://doi.org/10.1016/S0377-2217(00)00154-5} and references therein as well as \url{https://link.springer.com/article/10.1007/s11573-017-0885-1}; not sure what to make of this one \url{https://link.springer.com/article/10.1007/s11573-020-01004-x}

Have a bit more on the WTMBT and the possibility to derive scenario-based MRIOTs from it.

\addcontentsline{toc}{section}{Appendix B - old discussion notes}
\section*{Appendix B - old discussion notes}
0. Both RCOT and WTM as a game between man and nature (Morton \url{https://www.jstor.org/stable/2549610?origin=crossref&seq=13}); relationship of these games (minimax) to precationary principle (maximin) as per \url{https://onlinelibrary.wiley.com/doi/epdf/10.1111/j.1467-9760.2006.00237.x}; be careful as to the interpretation of the principle (recall Edgar's comment on (un)employment in such models); refer to \url{https://benjaminbrooks.net/downloads/brooksreny_equivalence.pdf} and \url{https://link.springer.com/article/10.1007/s00182-012-0328-8} and the printed paper (was it this one? \url{https://onlinelibrary.wiley.com/doi/abs/10.1111/j.1477-9552.1969.tb01357.x}); plus link potentially to tragedy of commons (perfectly mobile factors as "common pool resources")? \\
0x. Distinction between tables in monetary, physical, or hybrid units. Which constructs can be applied? Interpretation of price model as prices vs price indexes.\\
1. Role of comparative advantage - both models are almost identical in the sense that in both of them an alternative way of producing something is available; the interpretations of this mechanism are however very different \\
2. A different interpretation of comparative advantage: Compare Duchin's model with ten Raa's/Shestalova's. Formulate Duchin's model as an MCP to outline the differences (assumed baseline factor prices vs final demand volume variable (ten Raa postulates that knowledge about consumer preferences is necessary, e.g. here \url{https://journals.openedition.org/oeconomia/1833}) + ten Raa's sector-specific factor prices). All boiling down to a different (future sustainability) world view where all factors cost something (this could be the general theme of the paper); ten Raa maximises profit in Leontief's trade model while Duchin maximises profit less scarcity payments \url{https://pure.uvt.nl/ws/portalfiles/portal/969574/neoclass.pdf}\\
3. Comparative advantage and NSTs: the role of number of factors, fully-employed factors, and shared vs tech-specific factors \\
4. Comparative advantage vs hypothetical extraction: an (additive) inverse problem? / two sides of the same coin / a primal and dual perspective; comp-adv models assume an additional technology to be available whereas HEM assumes that a chosen technology is not available (with also here very different purposes: disaster impact and material pressures vs trade and tech choice) \\
5. Comparative advantage models and ecological economics: describe the role of substitution, with an emphasis on factor substitution and weak sustainability; comparative advantage models are not proper ecolecon models but fall into the pragmatic environmental econ corner \url{https://www.clivespash.org/wp-content/uploads/2015/04/2013_Spash_EE_Shallow_or_Deep.pdf}; is the comp-adv relationship a point to strengthen a classical interpretation of IO or at least these models here? (ten Raa's model of comp-adv. is even neoclassical...); refer also to Baumgärtner's joint production paper

Perhaps also a note on the similarity of RCOT and WIO-LP?
Link to Ayres\&Kneese; is material balance principle adhered to (in any unit-formulation?)
Does it make sense to connect to Daly's qualitative pollution input-output model?
Could Dyckhoff's paper be interesting for discussion on bads? \url{https://publications.rwth-aachen.de/record/949107/files/949107.pdf}

Highlight that all of above interpretations refer to RCOT / WTM as originally presented by Duchin; joint production concerns can then of course also be interpreted via SU-RCOT

It might be useful to explore the role of commodity prices more e.g. with reference to Guillaume's paper \url{https://www.sciencedirect.com/science/article/pii/S0921800916302269#s0005}

I think I should highlight the role of price homogeneity even more, e.g. with this paper by Algarin \url{https://onlinelibrary.wiley.com/doi/10.1111/jiec.12502#jiec12502-bib-0015}. Although not mentioned there, be reminded of Dietzenbacher's comment \url{https://sci-hub.se/https://www.sciencedirect.com/science/article/abs/pii/S0921800905002119?via%3Dihub#bib13} on sub-sector flows and according prices, thus pointing to the aggregation problem outlined by Guillaume.

Also look into the difference of factors being priced or not priced. How does it affect the model outcome if >=1 factors are not priced but included as factor constraints, and even more so what happens if this unpriced factor becomes active (does a scarcity rent occur?)?

Point out that RCOT-WTM fits generally very well to model not only planetary boundaries but doughnut economy (put into section on matrix games?). Just a difficulty to find social categories that are dependent of activity levels. Anyway, just highlight and give also example of needs-based perspective (at least by disaggregating final demand by need-categories)

Note that factors don't have to have an ex-ante price. In that case, it simply means that this factor is of no importance to society (no precautionary action!). That factor might however have to be priced if in scarce supply - then, the price would be determined purely by the scarcity rent. That hybrid nature of the price allows to corroborate the theory of comparative advantage, where also originally unpriced factors may change international trade - simply by transgressing biophysical boundaries.

On the topic of factor mobility: It seems that the exclusion of non-tradeables (matrix J) in ten Raa is dealt with very similiarly \url{https://www.tandfonline.com/doi/epdf/10.1080/09535319100000012?needAccess=true}. Include also the very recent paper by Dilekli \url{https://www.tandfonline.com/doi/full/10.1080/09535314.2023.2272213}

HO model cannot be used for deriving commodity prices as it assumes given commodityprices. This model differs therefore starkly from the others. Unless one starts with a monetary SUT/IOT, it is (I think) virtually impossible to determine commodity prices. The focus of the model is thus clearly on the actual output under imposed/given prices; isn't that problematic even in trade considerations? It certainly is when it comes to national modelling, because... how are the prices determined/made up?

ten Raa writes that exceptional economies are not interesting because they can "either produce nothing or anything". In RCOT, I think such anything-economies are not possible - at one point an infeasibility always kicked in. But NEO is pretty close to that - although he uses the NEO problem to show non-substitution; don't quite understand... , \url{https://www.sciencedirect.com/science/article/abs/pii/S0022053185710629?via%3Dihub}

Johansen writes that no additional assumptions are imposed on a standard linear IO model. But what really is a standard IO model? Where is it described that factor prices have to be endogenous?

Need to have a separate section on economic rent and what is meant by it, i.e. which production factors can I actually include at all where the concept of rent becomes intuitive?

This LCA review on technical substitutability could be interesting to add. They look at how the substitution of products (e.g. primary vs secondary metals) is modelled in LCA. Can product substitution be modelled with RCOT? \url{https://www.sciencedirect.com/science/article/pii/S0956053X23005482?dgcid=author}

Does RCOT follow the "biophysical theory of  value", as discussed by Patterson? Or how does it relate to it? That is, can I classify the RCOT dual in some way? It seems that it is pretty close to Costanza and Hannon's approach... \url{https://doi.org/10.1016/S0921-8009(97)00166-3} + when combined with waste, is it similar to my suggested model? Also, maybe RCOT requires the explicit definition of a numeraire?

Maybe it's easier to classify the role of value/price in RCOT using figure 2 and table 1 from here \url{https://www.sciencedirect.com/science/article/pii/S0921800922000441} Also this article might be helpful: \url{https://www.sciencedirect.com/science/article/pii/S0921800919308651}

Have a good deal about the early work of LCA-LP and how RCOT/WTM relates to it

If subsidies and taxes are included as factors: Would subsidies (or negative TLS) result in a joint-produciton setting? That is, in IO-RCOT, only one element per column is positive (in the constraint matrix) whereas in SU-RCOT, several can be. Factors are all negative (because of inequality sign), but subsidies would be positive. Or can that be alleviated by only saying that subsidies have a negative sign but that the corresponding factor price is +/-1?

Perhaps a useful intro to factor mobility with factor examples: \url{https://www.sciencedirect.com/science/article/pii/S0166046205000074?casa_token=wTqO3r8ABRQAAAAA:BwytLvQMs_zPrgQbELWkEUPYzZ0TcBGdPUm3nuxLlrrcPBdkdibJH3YKNS3h1E9RZWdVBJe6bQ}

Recent paper on factor mobility in WTM by Dilekli et al. \url{https://www.tandfonline.com/doi/full/10.1080/09535314.2023.2272213}

\nolinenumbers
\newpage
\newrefcontext[sorting=nyt] % sort the paper by name, year, title
\printbibliography[title = References in appendix]

\end{refsection}
\end{document}


