\documentclass{article}
\usepackage{packages}
\begin{document}

\import{./}{title}

\frontmatter
\tableofcontents

\mainmatter
\linenumbers

{\huge{\multiTitle}}

\section*{ABSTRACT}
\emph{To be filled in later}.

Key points: A review of literature using or describing the family of RCOT/WTM models.

\begin{refsection}

\section{Introduction}

Soon 20 years since publication of WTM paper.

The popularity of both RCOT and WTM models (in terms of applications) warrants for an overview and investigation of certain characteristics. We therefore review the existing literature in the first part of this review and classify the spectrum of RCOT/WTM models as well as point to potential research avenues in the second part.

One topic that garnered great interest in economics during the 1950s-60s but faded away afterwards was the problem of choosing among multiple technology alternatives. This topic was dealt with theoretically under the umbrella term of \textit{choice of technique}, which brought forth various non-substitution theorems. On the empirical side, \textcite{carter1970} examined structural changes in the US economy over 30 years or so and looked specifically at technology substitutions over time. \textcite{duchin_2011} generalised Carter's model into the \textit{rectangular choice of technology} model, or RCOT.

RCOT is a linear program (LP) that allows to choose among multiple technology alternatives with homogeneous outputs. It has been theoretically explored in multiple contributions \parencite{duchin_2017,steenge_2019,duchin_2012} and applied in various studies (refs). RCOT minimises factor use while considering factor constraints and satisfying final demand.

Driven by rising environmental literacy, another topic sparked great interest at the same time. Researchers from various disciplines began to engage in efforts that came to be known as \textit{world modelling} \parencite{fontela_2004}. \textit{The Limits to Growth} by \textcite{meadows_1972} enjoyed possibly both most popularity and criticism. In the input-output community, it was Leontief's world model that became well-known \parencite{leontief_1970,leontief_1977}. It is this latter model that \textcite{duchin_2005} generalised into a comparatively simple LP.

Have a split lit overview: some very old stuff (as a precursor to both WTM and IO-RCOT; in History section), and then the work that built on IO-RCOT and WTM respectively (in Review section). Have a further distinction for LCA and IO applications (only for RCOT relevant?)

Have the main review as a semi-systematic review with more detail in the SI. Have the review grouped by applications (according to individual planetary boundaries) as well as methodological extensions. Generally, it shall point out point out commonalities, differences, characteristics, mechanisms etc.


\section{Historical background}

Start from different thread and show how they converge: LP theory, environmental concerns, scenario modelling, IO developments (on the international stage), LP applications for environmental analysis

Describe the origin (both historically and conceptually) of RCOT+WTM -> world modelling + comparative advantage (Ricardo)

Both models are based on the same underlying mechanism, comparative advantage, yet are used for very different purposes. Let us now examine how we got to this point.

Refer to Kantorovich's original paper on LP (Kantorovich shared Leontief's view that mathematicians and production workers should collaborate, see conclusion b), where he even argues for the minimisation of scrap, which could provide a nice link to Duchin's work \url{https://www.jstor.org/stable/2627082}. See also Koopmann's comment \url{https://www.jstor.org/stable/2627081}

Give also an overview of the earlier literature on LP-based environmental analysis, including in the LCA space, e.g. the work by Azapagic and Clift and especially by Freire \url{https://link.springer.com/article/10.1007/PL00013340} (which contains many nice references, especially Azapagic!) and Dyckhoff (\url{https://doi.org/10.1016/S0377-2217(00)00154-5} and references therein) as well as \url{https://link.springer.com/article/10.1007/s11573-017-0885-1} as well as WIO-LP; not sure what to make of this one \url{https://link.springer.com/article/10.1007/s11573-020-01004-x}. Reference to early activity analysis (Koopmanns, Baumol etc) should be given so as to show that tradition. Regarding environmental analysis, it might be an idea to point to the Ayres-Kneese paper where they present essentially an EE-IO system but cannot solve it (the Ayres-Kneese orientation may fit well with the production function idea in pt II).

Also, don't forget to refer to Heijungs-Suh who described the difficulties of rectangular outlays (both for single production and joint production).

Relevant for WTM (and specifically factor mobility)? \url{https://www.jstor.org/stable/1907438} Does WTM actually reach back to Arrow and his reference to an article by Chenery?

\section{Model overview}

Notation. Plain introduction to the primal and dual models of both WTM and RCOT.

As we will describe later, the WTM and RCOT depend on the same underlying mechanism which is based on the principle of comparative advantage.
The most parsimonious formulation is the WTM/RCOT model presented in \textcite{duchin_2012}, where the primal reads as follows:

\begin{mini}
    {\bm{x}^{*}}{\sum_{r} \bm{\pi}^{\prime}_{r} \bm{F}^{*}_{r} \bm{x}^{*}_{r}}{\label{eq:IOT_RCOT_quant}}{Z_{IO}=}
    \addConstraint{\sum_{r} (\bm{I}^{*}_{r} - \bm{A}^{*}_{r}) \bm{x}^{*}_{r}}{= \sum_{r} \bm{y}_{r}}{}
    \addConstraint{\bm{F}^{*}_{r} \bm{x}^{*}_{r}}{\le \bm{f}_{r}}{\quad \forall r}
    \addConstraint{\bm{x}^{*}_{r}}{\ge \bm{0}}{\quad \forall r,}
\end{mini}

and the dual as:

\begin{maxi}
    {\bm{p}, \bm{r}_{r}}{\sum_{r} \bm{y}^{\prime}_{r} \bm{p} - \bm{f}^{\prime} \bm{r}}{\label{eq:IOT_RCOT_price}}{W_{IO}=}
    \addConstraint{(\bm{I}^{*}_{r} - \bm{A}^{*}_{r})^{\prime} \bm{p} - {{\bm{F}}^{*}_{r}}^{\prime} \bm{r}_{r}}{\le {{\bm{F}}^{*}_{r}}^{\prime} \bm{\pi}_{r}}{\quad \forall r}
    \addConstraint{\bm{p}, \bm{r}_{r}}{\ge \bm{0}}{\quad \forall r.}
\end{maxi}

The model presented in equation \ref{eq:IOT_RCOT_quant} (and \ref{eq:IOT_RCOT_price}) assumes region-specific resources and, if a net surplus in the material balance is permitted, includes a benefit-of-trade constraint:

\begin{equation}
    \Bar{\bm{p}_{r}}^{\prime}(\bm{I}^{*}_{r} - \bm{A}^{*}_{r}) \bm{x}^{*}_{r} \le \Bar{\bm{p}_{r}}^{\prime} \bm{y}_{r} \forall r
\end{equation}

If resources are instead pooled globally, the \emph{one-region} world trade model results. This is done by leaving aside the benefit-of-trade constraint and replacing the factor constraint equation of the primal model by:

\begin{equation}
    \sum_{r=1}^{n} \bm{F}^{*}_{r} \bm{x}^{*}_{r} \le \sum_{r=1}^{n} {\bm{f}_{r}}
\end{equation}

The dual model changes accordingly (see appendix).

To account for bilateral trade, \textcite{stromman_2006} reformulated the original world trade model such that the material balance reads:

\begin{equation}
    (\bm{I}^{*}_{r} - \bm{A}^{*}_{r}) \bm{x}^{*}_{r} - \sum_{r \neq s}^{n} \bm{e}_{rs} + \sum_{r \neq s}^{n}  (\bm{I}^{*}_{r} - \bm{T}^{*}_{r}) \bm{e}_{rs} \ge \sum_{r=1}^{n} {\bm{y}_{r}} \forall r
\end{equation}

where in contrast to the original material balance now the matrices are rearranged so that transport sectors are located at the bottom-right.

Have a bit more on the WTMBT and the possibility to derive scenario-based MRIOTs from it.

Quickly recap similarity of RCOT and WIO-LP \parencite{kondo_nakamura_2005}

While RCOT is a more parsimonious and general version of Carter's square COT model, WTM is more or less directly based on square COT in the case of a free trade world.


\begin{itemize}
    \item characteristics
    \item examination of feasibility criteria, non-substitution, and shared factors
    \item Lagrange multipliers
    \item justification of ixi/cxc tables
    \item differentiation to sector disaggregation
    \item perhaps something on the altered calculation of measures such as total factor productivity (and possibly also of circularity indicators \url{https://onlinelibrary.wiley.com/doi/10.1111/jiec.13446}. I suppose that all of that would simply be scenario-based and, if not directly, could be derived at least when doing the WTMBT (not sure, though))
\end{itemize}

\section{Review}

Review past applications (list here \url{https://ntnu.box.com/s/6ycrtybl99smyoopu18hmtkw2jb0jons}) and how they might have extended the model framework. Describe how both WTM and RCOT are generally to be applied (scenario-modelling etc) and then actual published applications.

Tag studies based on which planetary boundaries they can be related to as well as what methodological advancement they bring.

We review the literature that sprang off from the original RCOT and WTM publications. In particular, our review includes all studies citing these original publications or any of the following studies: \textcite{stromman_2006} and \textcite{kätelhön_2016}. We consider these latter two studies additionally, because publications citing them do not always refer back to the original RCOT/WTM publications.

Other extensions are far away from original RCOT but relevant to mention (probably in History section or in a discussion), e.g. dynamic model by Hung \url{https://onlinelibrary.wiley.com/doi/10.1111/jiec.13331#jiec13331-bib-0061}. This review of CLCA methods (incl. TCM) may also be relevant for IO applications \url{https://onlinelibrary.wiley.com/doi/10.1111/jiec.12983}, even if just to mention that other approaches exist.

\section{Conclusion}

\dots


\newrefcontext[sorting=nyt] % sort the paper by name, year, title
\printbibliography[heading = bibintoc] % 'bibintoc' inserts our bibliography into the table of contents

\end{refsection}

%%%%%%%%%%%%%%
% Appendix
% Inserting appendix with separate settings
\newpage
\setcounter{page}{1}
\renewcommand{\thepage}{A-\arabic{page}}
\linenumbers*
\addappendix

%Reset numbering of tables and equations in appendix, starting with A.
\renewcommand{\thetable}{A.\arabic{table}}
\setcounter{table}{0}
\renewcommand{\theequation}{A.\arabic{equation}}
\setcounter{equation}{0}

\begin{refsection}
\addcontentsline{toc}{section}{Appendix A}
\section*{Appendix A}
\lipsum[1-2]

\nolinenumbers
\newpage
\newrefcontext[sorting=nyt] % sort the paper by name, year, title
\printbibliography[title = References in appendix]

\end{refsection}
\end{document}


