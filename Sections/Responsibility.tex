\subsubsection{Shared responsibility for factor exchanges}

Next, we want to establish a connection between the treatment of factor mobility in the present framework, the concept of shared responsibility, and the Coase theorem. The literature on shared responsibility addresses the problem of how to attribute factor exchanges equitably to actors along supply chains \parencite[see for example][]{lenzen_2007}. The Coase theorem \parencite[based on][]{coase_1960} postulates that, at zero transaction costs, the allocation of factors ($\bm{R} \widehat{\bm{x}}$ in our notation) is Pareto efficient and independent of the allocation of property rights (or structure of entitled factor mobility $(\bm{R} \ast \bm{P})$, in our context). While the consequences at the individual level may sometimes be perceived as unfair, thus questioning the optimality of factor allocation and making it tempting to attribute proportionally greater responsibility to a few selected actors, this view changes at the grander scale: due to its interconnectedness, Steenge (\textbf{REF: Some old book chapter by Steenge}) argues, the entire economy is effectively responsible for the factor exchanges occurring within it, such that each sector can be attributed a proportional share of the blame, provided that the economy is indecomposable. This insight is to be supplemented with the result by \textcite{amores_2014} that factor mobility increases allocative efficiency.

Taken together, this means that for two situations of differently allocated property rights, one may be more efficient than the other, even though the factor allocation in each of them is Pareto efficient. Put differently: An economy employs the most efficient combination of available technologies in any setting of constrained or unconstrained factor mobility and constrained factor availability, but the case of perfect factor mobility spares more resources while achieving a higher utility than any other setup. It becomes then interesting to consider the reasons for (partial) factor immobility. When a factor cannot move between sectors due to structural conditions, e.g. biophysical limits to land conversion, the originally chosen factor allocation remains in any case optimal. When, however, a factor cannot move because of arbitrary or otherwise easily removable limitations, e.g. sectoral emission budgets, the system's allocative efficiency is put into question: if the respective factor can be made mobile without incurring transaction costs, the factor allocation was not optimal and disproportionally greater responsibility may be attributed to individual actors; if no such adjustment is possible or only at non-zero transaction costs, the factor allocation was optimal and the responsibility may be split proportionally.

How does that look when a sector can choose between multiple technology alternatives? Due to the nature of IO-RCOT as an LP, we know that it minimises factor costs and as such aims at an optimal allocation of factor exchanges. Transaction costs are not explicitly represented. If all factors are perfectly mobile between technologies, the system's allocation of factors is optimal, irrespective of the occurrence of substitution between technology alternatives. If a factor is limited in its mobility between a given sector's technology alternatives or in its economy-wide mobility, the allocation's optimality depends yet again on the underlying reason for the respective factor's reduced mobility.

All these considerations are, of course, subject to the limitations posed by the choice of time horizon discussed earlier. Realistically, removing boundaries to a factor's mobility would incur transaction costs, thus putting the factor allocation's (and system's) optimality into question. At the same time, introducing new technology alternatives may require an arbitrary choice as to the proportion and speed of mobility of its required factors. Either way, it seems that IO-RCOT in its current form is not entirely adequate to account for losses due to transaction costs, modification of factor mobility, overcoming of factor stickiness, and responsibility attribution, all of which is necessary to evaluate if the system's technology choice is optimal over time.