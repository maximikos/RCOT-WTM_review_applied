\subsubsection{Shifting production factors}

When factors are shared by multiple technologies, both existing and/or alternative ones, it is worth considering the aspect of factor mobility. Although the term is most commonly applied in the context of trade economics, we use it here to refer to the (im)mobility of production factors between industries. Why might this be relevant in the RCOT context? 

Let us first consider the empirical base setting without alternative technologies ($t = k$) and where the employed factors are assumed to be the actual factor limits $\bm{f}^{\circ} \le \bm{f}$; i.e. denote for notational simplicity $\bm{\Phi} = (\bm{F} \lor \bm{R} \widehat{\bm{x}^{\circ}})$ (depending on if $i \times i$ or $c \times c$ setting) such that $\bm{\Phi i} = \bm{\phi} = \bm{f}^{\circ}$ and where $\bm{x}^{\circ}$ is the empirically observed total output while $\bm{x}$ is the total output resulting from the LP. Further, let us introduce the $p \times k$ factor mobility matrix $\bm{P}$ which shall define which factors can move between which sectors. Let $\bm{R} \ast \bm{P}$ be the Khatri-Rao product, or block-wise Kronecker product of matrices $\bm{R}$ and $\bm{P}$. Then, in the case of perfect factor mobility, one may represent the system's factor constraints as $(\bm{R} \ast \bm{P}) \bm{x} = \bm{R} \bm{x} \le \bm{f}^{\circ} = (\bm{R} \ast \bm{P}) \bm{x}^{\circ} = \bm{R} \bm{x}^{\circ}$, where $p = v + r$ and where the rows of $\bm{P}$ consist of row unit vectors $\bm{i}'$, one for each factor. In contrast to that, perfect immobility of all factors may be represented as $(\bm{R} \ast \bm{P}) \bm{x} \le (\bm{R} \ast \bm{P}) \bm{x}^{\circ}$, where $p = k \times (v + r)$ and where each $k \times k$ row block contains an identity matrix $\bm{I}$. Alternatively, one may express the latter case as $\bm{R} \widehat{\bm{x}} \le \bm{\Phi} = \bm{R} \widehat{\bm{x}^{\circ}}$ where the inequality applies now element-wise. Perfect immobility of all factors is thus the situation when all considered factors are technology-specific.

The original formulation by \textcite{duchin_2011} models primarily perfectly mobile factors, i.e. where all factors can move freely between industries as per the factor constraint $\bm{\phi}^{*} = \bm{R}^{*} \bm{x}^{*} \le \bm{f}$ shown in model \ref{eq:IOT_RCOT_quant}\footnotemark{}. One may easily represent this constraint with the same formalism as just seen, now only extended for the respective technology alternatives: $(\bm{R}^{*} \ast \bm{P}^{*}) \bm{x}^{*} = \bm{R}^{*} \bm{x}^{*} \le \bm{f}$ where $\bm{P}$ is of shape $(v + r) \times t$.

In another context and for another model, the case of perfect factor immobility was highlighted by, among others, \textcite{tenRaa_1994,shestalova_2019}; it describes a situation when all factors cannot or shall not move between industries. As seen above, this is straightforward to model when the available factors are fully employed by the given technologies where no additional technology alternatives are present. However, once our original system turns into the rectangular one $(\bm{A}^{*}, \bm{I}^{*}, \bm{R}^{*})$, we struggle to define a factor constraint equivalent to the one above because of the dimension mismatch between left-hand and right-hand sides (LHS and RHS) of the inequality. We suggest two ways to deal with this issue and explain them, for the sake of generality and the present context of technology choice, for the case when we set the allowed factor use to the actually available factor endowments $\bm{f}$.

\footnotetext{One may speak in this case of economy-wide factor pools.}

Targeting the RHS of the factor constraint, the first approach is to allocate each available factor according to some rule to each technology, both old and new, thus creating technology-specific factor budgets. Let us introduce the $q \times 1$ block-wise factor allocation vector $\bm{\eta}^{*}$ for that purpose where each block sums to 1 and be $\bm{\eta}^{*} \ast \bm{f}$ the Khatri-Rao product of $\bm{\eta}^{*}$ and $\bm{f}$. If we allocate each factor to each technology, $\bm{\eta}^{*}$ be of row-dimension $q = (v + r) \times t$ and where each element of each $t \times 1$ row block allocates a specified share of the total factor budget to each technology. The resulting factor constraint for perfect factor immobility is $\bm{R}^{*} \bm{x}^{*} \le \bm{\eta}^{*} \ast \bm{f}$. If one were to apportion factors to technology groups instead of only individual technologies, the row dimension of $\bm{\eta}^{*}$ would change accordingly to $q = (v + r) \times \tau$ where $k \le \tau \le t$; one will have to adjust the LHS appropriately.

Alternatively, manipulating the LHS of the factor constraint, i.e. $\bm{\phi}^{*}$, one may choose to aggregate the to-be-modelled factor use of specified technologies such that factors are mobile between $\rho$ chosen sets of technologies, here between a technology and its alternative, but immobile between the rest. In this case, one may formulate the factor constraint similar to earlier as $(\bm{R}^{*} \ast \bm{P}^{*}) \bm{x}^{*} \le \bm{f}$ but where $\bm{P}^{*}$ is now of shape $((v + r) \times (\rho + 1)) \times t$ and where each $(\rho + 1) \times t$ row block sums along its rows to unity $\bm{i}'$. If one aggregates the factor use of more than the originally dominant technologies and their respective alternatives such that the row dimension of $\bm{\phi}^{*}$ becomes less than $(v + r) \times k$, one has to adjust the RHS of the factor constraint accordingly. Clearly, the reduction of factor mobility to selected technology sets does not impose perfect factor immobility as strictly on the system as the previous approach, and reallocation of production enabled by shared factors between the chosen technologies becomes a necessary possibility.

These two approaches allow to impose (almost) perfect factor immobility between all technologies available in the extended system. This stands in contrast to the perfect mobility of all factors seen earlier and which increases allocative efficiency \parencite{amores_2014}. A middle-ground between both extremes can be found when one chooses to impose selective, partial mobility of factors; selective in the sense that multiple technologies that are not necessarily only alternatives to each other share factor pools; and partial in the sense that multiple technology groups may share common factor pools to differing degrees. Both can be easily accommodated for with the allocation and mobility matrices introduced above. For the former, by allowing the technology sets $\rho$ to be filled with any technology combination (not only the dominant technology and its alternatives being grouped together), and for the latter by allowing elements of the $\bm{P}^{*}$-matrix to take on values other than zero or one as long as the column sums of each row block equal unity, while at the same time ensuring that the available factors are apportioned accordingly through a sensible use of $\bm{\eta}^{*}$. Additionally, one may allow for different allocation patterns for different factors, instead of the homogeneous way presented here. Importantly, however, the row dimensions of LHS and RHS of the factor constraint must match. It is worthwhile mentioning at this point that any deviation away from assuming perfect factor mobility as expressed in the IO-RCOT model may require the adjustment of not only the factor constraint but also other parts of the primal and dual; for example, one may choose to formulate the primal's objective function as \{$\underset{x^{*}}{\rm minimize} \quad \bm{i}' (\Bar{\bm{\pi}}' \ast {\bm{P}^{*}}^{\prime}) (\bm{R}^{*} \ast \bm{P}^{*}) \bm{x}^{*} $\} in the case of perfect factor immobility. Moreover, while any of these approaches may help to model between which sectors the factors can move and to which degree, we have not set any measure of their stickiness, i.e. how easily factors move from one sector to another.\footnotemark{}

\footnotetext{Somewhat in relation to sector-specific factor constraints, one may also choose to consider sector-specific factor prices $\Bar{\bm{\Pi}}$ (or $\Bar{\bm{\Pi}}^{*}$ if technology alternatives are added). In that case, both primal and dual would have to be adjusted accordingly, e.g. the primal's objective function would turn into $\bm{i}' \Bar{\bm{\Pi}}^{* \prime} \odot \bm{R}^{*} \bm{x}^{*}$, where $\odot$ represents the element-wise Hadamard product.}

To counter this problem, and thus widen the problem space, one could try to implement some measure of inertia. This could be done by including an additional objective function, thus resorting to multi-objective optimisation \parencite[for a review of multi-objective LP in IO, see ][]{oliveira_2014}. This additional objective function could, similar to the well-known RAS-procedure, minimise the system difference, i.e. the difference between the empirical base state (with or without active factor constraints and without alternative technologies) and the new state where either new technology alternatives were introduced, factor constraints became active, or both. The new objective function would hence be \{$ {\rm minimize} \quad (h_{1}(x^{*}), h_{2}(x^{*}))$\}, where $h_{1}(x^{*})$ is the original objective function and where $h_{2}(x^{*})$ sets the two system states $(\bm{S}^{*} \ast \bm{P}^{*}) \bm{x}^{*} \le (\bm{S} \ast \bm{P}) \bm{x}^{\circ}$ into relation, with the factor mobility matrices being of appropriate shape.

The exact shape of $h_{2}(x^{*})$ is, similar to any of the functions employed for the RAS-procedure \parencite{jackson_2004}, subject to individual considerations; then, when weighting the state difference, an implicit factor inertia is introduced (by trying to keep the differences between the systems small). As for the combination of both objective functions, generally, if an \textit{a priori} solution method is chosen, a certain preference between the two objective functions has to be defined - which serves as a proxy for determining the stickiness of factors. It should be noted that the economic interpretation of the LP is changed once the additional objective function is introduced, and that the formulation of a dual is thus also complicated. The possibility of dealing with the problem of factor mobility and stickiness is only insinuated here but deserves a more focused examination elsewhere, especially if one wants to model it explicitly.

As regards the factor mobility of capital goods, one may be moreover reminded of \citeauthor{carter1970}'s (1970, chapter 12) discussion of the embodiment hypothesis and putty-clay models which certainly applies to the more parsimonious RCOT framework as well. In fact, as IO-RCOT is formulated now, with perfectly mobile factors, it allows for both ex-ante and ex-post flexibility. Lock-in effects due to (partially) immobile factors implying ex-post rigidity are, however, an empirical reality which mandate the sensible modification of the factor constraint and an appropriate choice of time horizon depending on the purpose of the study.

As for that choice of time horizon: if the modeller's purpose is to evaluate only the \textit{hypothetical} competitiveness (or efficiency) of emerging, obsolete, or hypothetical vs dominant technologies, i.e. if technology $t_{A}$ would be more efficient than technology $t_{B}$ in terms of factor exchanges and factor cost reduction, then the limitations posed by factor immobility can be disregarded and the analysis can be undertaken for the chosen accounting period in past, present, and future.\footnotemark{} If the purpose is, however, to construct a future scenario to assess if an emerging (or entirely hypothetical) technology could \textit{actually} substitute a dominant technology, factor (im)mobility and stickiness become important: if the time horizon is too close to the present, it is questionable whether certain factors like labour force and capital goods could shift or be set up fast enough to allow substitution; if the modelled time period lies far in the future, the modelling may perhaps not be limited by the problem of factor stickiness but rather by the availability of sensible data for the technological recipes of both originally dominant and new technologies. On a side note, if the modeller's purpose is to perform either of the above for the case where the supposedly new technology is already incumbent, i.e. it is part of a dominant sector, the latter's technological recipe must be split and adjusted. Concluding, if the study's aim is not a hypothetical efficiency analysis, a more realistic, explicit modelling of factor mobility and stickiness is warranted, potentially requiring a dynamic model formulation.

\footnotetext{Provided appropriate data is available, one may choose to perform this analysis not only for the typical period of a year but any arbitrary period.}