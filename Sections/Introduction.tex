\section{Introduction}

Imagine an economy that is part of the world economy. In that economy, production and consumption activities take place, and at the intersection of that economy and the system it is embedded in

Start from different thread and show how they converge: LP theory, environmental concerns, scenario modelling, IO developments (on the international stage), LP applications for environmental analysis


This paper serves as a background paper for my SU-paper. It shall describe the RCOT, WTM, and WTM/RCOT models (including their extensions), point out commonalities, differences, characteristics, mechanisms etc.

Mention somewhere in intro that the popularity of both models (in terms of applications) warrants for an overview and investigation of certain characteristics

Perhaps include Samuelson's paper on Ricardian land \url{https://academic.oup.com/qje/article/73/1/1/1938462}

Refer to Kantorovich's original paper on LP (Kantorovich shared Leontief's view that mathematicians and production workers should collaborate, see conlcusion b), where he even argues for the minimisation of scrap, which could provide a nice link to Duchin's work \url{https://www.jstor.org/stable/2627082}. See also Koopmann's comment \url{https://www.jstor.org/stable/2627081}

Give also an overview of the earlier literature on LP-based environmental analysis, including in the LCA space, e.g. the work by Azapagic and Clift and especially by Freire \url{https://link.springer.com/article/10.1007/PL00013340} (which contains many nice references) and WIO-LP. Reference to early activity analysis (Koopmanns, Baumol etc) should be given so as to show that tradition. Regarding environmental analysis, it might be an idea to point to the Ayres-Kneese paper where they present essentially an EE-IO system but cannot solve it (the Ayres-Kneese orientation may fit well with the production function idea).

Quite a few papers also start from Kätelhöhn's model but do not give reference to D\&L (e.g. \url{https://pubs.acs.org/doi/10.1021/acssuschemeng.7b00631} or Steubing \url{https://link.springer.com/article/10.1007/s11367-015-1015-3} or Bachmann on plastic \url{https://www.nature.com/articles/s41893-022-01054-9}). Other extensions are far away from original RCOT but relevant to mention, e.g. dynamic model by Hung \url{https://onlinelibrary.wiley.com/doi/10.1111/jiec.13331#jiec13331-bib-0061}. This review of CLCA methods (incl. TCM) may also be relevant for IO applications \url{https://onlinelibrary.wiley.com/doi/10.1111/jiec.12983}, even if just to mention that other approaches exist.

Maybe have a split lit overview: some very old stuff (as a precursor to both WTM and IO-RCOT), and then the work that built on IO-RCOT and WTM respectively. Have a further distinction for LCA and IO applications (only for RCOT relevant?)

Have a semi-systematic review with more detail in the SI. Have the review grouped by applications (according to individual planetary boundaries) as well as methodological extensions.

Also, don't forget to refer to Heijungs-Suh who described pretty much SU-RCOT in words and the difficulties of rectangular outlays.

RCOT is more like an analysis technique than a causal model to explain substitution. That is, the cause for technical change is not explained, only the result of it.
WTM is a model that shows which trade relationships would be present in an idealised world (according to comparative advantage).

One topic that garnered great interest in economics during the 1950s-60s but faded away afterwards was the problem of choosing among multiple technology alternatives. This topic was dealt with theoretically under the umbrella term of \textit{choice of technique}, which brought forth various non-substitution theorems. On the empirical side, \textcite{carter1970} examined structural changes in the US economy over 30 years or so and looked specifically at technology substitutions over time. \textcite{duchin_2011} generalised Carter's model into the \textit{rectangular choice of technology} model, or RCOT.

RCOT is a linear program that allows to choose among multiple technology alternatives with homogeneous outputs. It has been theoretically explored in multiple contributions \parencite{duchin_2017,steenge_2019,duchin_2012} and applied in various studies (refs). RCOT minimises factor use while considering factor constraints and satisfying final demand.

Driven by rising environmental literacy, another topic sparked great interest at the same time. Researchers from various disciplines began to engage in efforts that came to be known as \textit{world modelling} \parencite{fontela_2004}. \textit{The Limits to Growth} by \textcite{meadows_1972} enjoyed possibly both most popularity and criticism. In the input-output community, it was Leontief's world model that became well-known \parencite{leontief_1970,leontief_1977}. 

The structure of the paper is as follows: after providing an overview of the RCOT model as formulated by \textcite{duchin_2011} in section x, we present a supply-driven counterpart and variations of it in terms of how factor exchanges are represented (section y). We then illustrate these models with a numerical example and point out key differences between the different variations (section z), after which we conclude (section zz).

Relevant for WTM (and specifically factor mobility)? \url{https://www.jstor.org/stable/1907438} Does WTM actually reach back to Arrow and his reference to an article by Chenery?
Relevant for substitution of Leontief system generally \url{https://www.sciencedirect.com/science/article/pii/0024379568900025}
Export and import prices become important in WTM (in addition to sector-specific prices and input vs output prices)
Does cross-hauling make a difference (or can even be accounted for) in WTM?

What if factors are not priced exogenously? Does a scarcity rent arise?

Production function as from here: \url{https://www.jstor.org/stable/23859612?seq=11} and from van den Bergh \url{https://www.jstor.org/stable/3147065?}

definitely include WIO-LP when talking about RCOT

When it comes to an overview of/ reference for WIO, use this review paper by Towa \url{https://www.sciencedirect.com/science/article/pii/S0959652619342295?via%3Dihub} and this paper to refer to the most recent dynamic WIO model \url{https://pubs.acs.org/doi/full/10.1021/acs.est.2c09676}. Mention at a later stage that the limitations of IO-RCOT also affect every WIO model and that perhaps a SU-based model might be better (because fewer limitations). this is especially interesting concerning critical materials that are typically only represented as by-products of other metal production processes.

This overview of comparative advantage literature by Belloc suggests that more needs to be included in respective models (and thus also in RCOT/WTM); especially the role of institutions is missing - only factor endowments and technological differences do not explain actual trade patterns \url{https://onlinelibrary.wiley.com/doi/abs/10.1111/j.0950-0804.2006.00274.x}. This can easily be pointed to in the qualitative production functions.
I suppose that these institutional setting don't only have an effect on trade between countries but also between industries of the same country (cartels within a country could be one special form? next to obvious legislative differences - what about standards and laws for different industries?)
Could the institutional aspect be connected to the needs-perspective. I.e. different institutional settings may affect final demand groups differently?

I might want to use the production functions to point out opportunities for further research, e.g. to include institutions, waste etc
But I should also highlight that both models are only applicable under certain assumptions (global quasi-free trade, linearity etc.) and that it might be too limited for more sophisticated analyses (e.g. concerning dynamic de-growth analyses where disequlibria and feedback effects may occur)
Is the assumption and modelling of comparative advantage a normative dilemma? I.e. if world organisation were different would comparative advantage be proven wrong?

It would be great if I could suggest an abstract model extension that allows to include considerations of e.g. insitutional setting, circularity, temporality, spatial granularity, qualitative demand differences (needs!) etc

Having capital as endowments is a problem. Switching to a Sraffian perspective with capital as produced goods may be more applicable.
Also, cross-hauling is not accounted for, is it?

Somewhere I should talk about (Keynesian) effective demand and what that means in terms of justification for both models, and then formulate a model where demand and supply are mixed (some fd and some x on the RHS).
As for (exogenous) supply (constraints): these can either be defined as a constraint, or supply and fd may be swapped for some goods, or the supply constraint can be introduced when turning the LCP into an MCP (As per here \url{https://juliabook.chkwon.net/book/complementarity})

Does WTM only consider trade in intermediate goods?

In the neoclassical understanding, an economy is generally represented as the set $E_{1} = (\phi, V, U, e)$ (cf. Shestalova's book chapter and cited references therein). $E_{1}$ is further characterised by all prices, i.e. both commodity and all sorts of factor prices being endogenous. In contrast, in Duchin's original, sustainable development-oriented models, an economy is instead represented as $E_{2} = (\pi, \phi, I, A, e)$, where the input and output matrices are based on pseudo-single production and where an exogenous factor mark-up reflects the precautionary principle. Kätelhöhn et al. and Koslowski et al. have expanded Duchin's rectangular choice model to cover joint production instead, resulting in an economy described as $E_{3} = (\pi, \phi, V, U, e)$; we see that also here a hybrid factor price is employed, with $\pi$ being the exogenous portion.

Classify the model according to Rose's scheme

When reviewing the TCM-based papers, I should add that the WTM formalism can also be applied to TCM-based approaches as origin-destination pairs also play a role in LCA.

For desirable factor exchanges (e.g. employment), the factor constraint must/should be an equality. Does it thus make sense to distinguish between desirable and undesirable factors exchanges? Equalities for desirable ones, inequalities for undesirable ones?
Maybe propose to have actually three factor equations/inequalities (<=>); not only to distinguish between desirable and undesirable factor exchanges, but also between in and outputs of factors

Not sure if this PIK paper on a new CBA-CEA/ Welfare economics is relevant as background \url{https://www.degruyter.com/document/doi/10.1515/jbnst-2022-0022/html#j_jbnst-2022-0022_ref_025}

Maybe include earlier LCA-related work on LP models, such as Freire \url{https://link.springer.com/article/10.1007/PL00013340} and references therein (especially Azapagic!), or Dyckhoff \url{https://doi.org/10.1016/S0377-2217(00)00154-5} and references therein as well as \url{https://link.springer.com/article/10.1007/s11573-017-0885-1}; not sure what to make of this one \url{https://link.springer.com/article/10.1007/s11573-020-01004-x}

Have a bit more on the WTMBT and the possibility to derive scenario-based MRIOTs from it.