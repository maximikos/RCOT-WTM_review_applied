\section{Interpretations \& observations}

In this section, we want to further characterise the models and provide additional interpretations. We hope that other researchers may find this helpful in the case of them considering using any of the models or in terms of potential avenues for further research.

Interpret both models

0. Both RCOT and WTM as a game between man and nature (Morton \url{https://www.jstor.org/stable/2549610?origin=crossref&seq=13}); relationship of these games (minimax) to precationary principle (maximin) as per \url{https://onlinelibrary.wiley.com/doi/epdf/10.1111/j.1467-9760.2006.00237.x}; be careful as to the interpretation of the principle (recall Edgar's comment on (un)employment in such models); refer to \url{https://benjaminbrooks.net/downloads/brooksreny_equivalence.pdf} and \url{https://link.springer.com/article/10.1007/s00182-012-0328-8} and the printed paper (was it this one? \url{https://onlinelibrary.wiley.com/doi/abs/10.1111/j.1477-9552.1969.tb01357.x}); plus link potentially to tragedy of commons (perfectly mobile factors as "common pool resources")? \\
0x. Distinction between tables in monetary, physical, or hybrid units. Which constructs can be applied? Interpretation of price model as prices vs price indexes.\\
1. Role of comparative advantage - both models are almost identical in the sense that in both of them an alternative way of producing something is available; the interpretations of this mechanism are however very different \\
2. A different interpretation of comparative advantage: Compare Duchin's model with ten Raa's/Shestalova's. Formulate Duchin's model as an MCP to outline the differences (assumed baseline factor prices vs final demand volume variable (ten Raa postulates that knowledge about consumer preferences is necessary, e.g. here \url{https://journals.openedition.org/oeconomia/1833}) + ten Raa's sector-specific factor prices). All boiling down to a different (future sustainability) world view where all factors cost something (this could be the general theme of the paper); ten Raa maximises profit in Leontief's trade model while Duchin maximises profit less scarcity payments \url{https://pure.uvt.nl/ws/portalfiles/portal/969574/neoclass.pdf}\\
3. Comparative advantage and NSTs: the role of number of factors, fully-employed factors, and shared vs tech-specific factors \\
4. Comparative advantage vs hypothetical extraction: an (additive) inverse problem? / two sides of the same coin / a primal and dual perspective; comp-adv models assume an additional technology to be available whereas HEM assumes that a chosen technology is not available (with also here very different purposes: disaster impact and material pressures vs trade and tech choice) \\
5. Comparative advantage models and ecological economics: describe the role of substitution, with an emphasis on factor substitution and weak sustainability; comparative advantage models are not proper ecolecon models but fall into the pragmatic environmental econ corner \url{https://www.clivespash.org/wp-content/uploads/2015/04/2013_Spash_EE_Shallow_or_Deep.pdf}; is the comp-adv relationship a point to strengthen a classical interpretation of IO or at least these models here? (ten Raa's model of comp-adv. is even neoclassical...); refer also to Baumgärtner's joint production paper

Perhaps also a note on the similarity of RCOT and WIO-LP?
Link to Ayres\&Kneese; is material balance principle adhered to (in any unit-formulation?)
Does it make sense to connect to Daly's qualitative pollution input-output model?
Could Dyckhoff's paper be interesting for discussion on bads? \url{https://publications.rwth-aachen.de/record/949107/files/949107.pdf}

Highlight that all of above interpretations refer to RCOT / WTM as originally presented by Duchin; joint production concerns can then of course also be interpreted via SU-RCOT

It might be useful to explore the role of commodity prices more e.g. with reference to Guillaume's paper \url{https://www.sciencedirect.com/science/article/pii/S0921800916302269#s0005}

I think I should highlight the role of price homogeneity even more, e.g. with this paper by Algarin \url{https://onlinelibrary.wiley.com/doi/10.1111/jiec.12502#jiec12502-bib-0015}. Although not mentioned there, be reminded of Dietzenbacher's comment \url{https://sci-hub.se/https://www.sciencedirect.com/science/article/abs/pii/S0921800905002119?via%3Dihub#bib13} on sub-sector flows and according prices, thus pointing to the aggregation problem outlined by Guillaume.

Pack all of the material on factor mobility here.

The same for shared responsibility and Coase and Pigou.

Also look into the difference of factors being priced or not priced. How does it affect the model outcome if >=1 factors are not priced but included as factor constraints, and even more so what happens if this unpriced factor becomes active (does a scarcity rent occur?)?

Point out that RCOT-WTM fits generally very well to model not only planetary boundaries but doughnut economy (put into section on matrix games?). Just a difficulty to find social categories that are dependent of activity levels. Anyway, just highlight and give also example of needs-based perspective (at least by disaggregating final demand by need-categories)

Note that factors don't have to have an ex-ante price. In that case, it simply means that this factor is of no importance to society (no precautionary action!). That factor might however have to be priced if in scarce supply - then, the price would be determined purely by the scarcity rent. That hybrid nature of the price allows to corroborate the theory of comparative advantage, where also originally unpriced factors may change international trade - simply by transgressing biophysical boundaries.

On the topic of factor mobility: It seems that the exclusion of non-tradeables (matrix J) in ten Raa is dealt with very similiarly \url{https://www.tandfonline.com/doi/epdf/10.1080/09535319100000012?needAccess=true}. Include also the very recent paper by Dilekli \url{https://www.tandfonline.com/doi/full/10.1080/09535314.2023.2272213}

HO model cannot be used for deriving commodity prices as it assumes given commodityprices. This model differs therefore starkly from the others. Unless one starts with a monetary SUT/IOT, it is (I think) virtually impossible to determine commodity prices. The focus of the model is thus clearly on the actual output under imposed/given prices; isn't that problematic even in trade considerations? It certainly is when it comes to national modelling, because... how are the prices determined/made up?

On the topic of factor mobility: It seems that the exclusion of non-tradeables (matrix J) in ten Raa is dealt with very similiarly \url{https://www.tandfonline.com/doi/epdf/10.1080/09535319100000012?needAccess=true}.


ten Raa writes that exceptional economies are not interesting because they can "either produce nothing or anything". In RCOT, I think such anything-economies are not possible - at one point an infeasibility always kicked in. But NEO is pretty close to that - although he uses the NEO problem to show non-substitution; don't quite understand... , \url{https://www.sciencedirect.com/science/article/abs/pii/S0022053185710629?via%3Dihub}

Johansen writes that no additional assumptions are imposed on a standard linear IO model. But what really is a standard IO model? Where is it described that factor prices have to be endogenous?

Need to have a separate section on economic rent and what is meant by it, i.e. which production factors can I actually include at all where the concept of rent becomes intuitive?

This LCA review on technical substitutability could be interesting to add. They look at how the substitution of products (e.g. primary vs secondary metals) is modelled in LCA. Can product substitution be modelled with RCOT? \url{https://www.sciencedirect.com/science/article/pii/S0956053X23005482?dgcid=author}

Does RCOT follow the "biophysical theory of  value", as discussed by Patterson? Or how does it relate to it? That is, can I classify the RCOT dual in some way? It seems that it is pretty close to Costanza and Hannon's approach... \url{https://doi.org/10.1016/S0921-8009(97)00166-3} + when combined with waste, is it similar to my suggested model? Also, maybe RCOT requires the explicit definition of a numeraire?

Maybe it's easier to classify the role of value/price in RCOT using figure 2 and table 1 from here \url{https://www.sciencedirect.com/science/article/pii/S0921800922000441} Also this article might be helpful: \url{https://www.sciencedirect.com/science/article/pii/S0921800919308651}

Have a good deal about the early work of LCA-LP and how RCOT/WTM relates to it

If subsidies and taxes are included as factors: Would subsidies (or negative TLS) result in a joint-produciton setting? That is, in IO-RCOT, only one element per column is positive (in the constraint matrix) whereas in SU-RCOT, several can be. Factors are all negative (because of inequality sign), but subsidies would be positive. Or can that be alleviated by only saying that subsidies have a negative sign but that the corresponding factor price is +/-1?

Perhaps a useful intro to factor mobility with factor examples: \url{https://www.sciencedirect.com/science/article/pii/S0166046205000074?casa_token=wTqO3r8ABRQAAAAA:BwytLvQMs_zPrgQbELWkEUPYzZ0TcBGdPUm3nuxLlrrcPBdkdibJH3YKNS3h1E9RZWdVBJe6bQ}

Recent paper on factor mobility in WTM by Dilekli et al. \url{https://www.tandfonline.com/doi/full/10.1080/09535314.2023.2272213}